\PassOptionsToPackage{unicode=true}{hyperref} % options for packages loaded elsewhere
\PassOptionsToPackage{hyphens}{url}
%
\documentclass[]{article}
\usepackage{lmodern}
\usepackage{amssymb,amsmath}
\usepackage{ifxetex,ifluatex}
\usepackage{fixltx2e} % provides \textsubscript
\ifnum 0\ifxetex 1\fi\ifluatex 1\fi=0 % if pdftex
  \usepackage[T1]{fontenc}
  \usepackage[utf8]{inputenc}
  \usepackage{textcomp} % provides euro and other symbols
\else % if luatex or xelatex
  \usepackage{unicode-math}
  \defaultfontfeatures{Ligatures=TeX,Scale=MatchLowercase}
\fi
% use upquote if available, for straight quotes in verbatim environments
\IfFileExists{upquote.sty}{\usepackage{upquote}}{}
% use microtype if available
\IfFileExists{microtype.sty}{%
\usepackage[]{microtype}
\UseMicrotypeSet[protrusion]{basicmath} % disable protrusion for tt fonts
}{}
\IfFileExists{parskip.sty}{%
\usepackage{parskip}
}{% else
\setlength{\parindent}{0pt}
\setlength{\parskip}{6pt plus 2pt minus 1pt}
}
\usepackage{hyperref}
\hypersetup{
            pdftitle={Practica2},
            pdfauthor={Maite Piedra},
            pdfborder={0 0 0},
            breaklinks=true}
\urlstyle{same}  % don't use monospace font for urls
\usepackage[margin=1in]{geometry}
\usepackage{color}
\usepackage{fancyvrb}
\newcommand{\VerbBar}{|}
\newcommand{\VERB}{\Verb[commandchars=\\\{\}]}
\DefineVerbatimEnvironment{Highlighting}{Verbatim}{commandchars=\\\{\}}
% Add ',fontsize=\small' for more characters per line
\usepackage{framed}
\definecolor{shadecolor}{RGB}{248,248,248}
\newenvironment{Shaded}{\begin{snugshade}}{\end{snugshade}}
\newcommand{\AlertTok}[1]{\textcolor[rgb]{0.94,0.16,0.16}{#1}}
\newcommand{\AnnotationTok}[1]{\textcolor[rgb]{0.56,0.35,0.01}{\textbf{\textit{#1}}}}
\newcommand{\AttributeTok}[1]{\textcolor[rgb]{0.77,0.63,0.00}{#1}}
\newcommand{\BaseNTok}[1]{\textcolor[rgb]{0.00,0.00,0.81}{#1}}
\newcommand{\BuiltInTok}[1]{#1}
\newcommand{\CharTok}[1]{\textcolor[rgb]{0.31,0.60,0.02}{#1}}
\newcommand{\CommentTok}[1]{\textcolor[rgb]{0.56,0.35,0.01}{\textit{#1}}}
\newcommand{\CommentVarTok}[1]{\textcolor[rgb]{0.56,0.35,0.01}{\textbf{\textit{#1}}}}
\newcommand{\ConstantTok}[1]{\textcolor[rgb]{0.00,0.00,0.00}{#1}}
\newcommand{\ControlFlowTok}[1]{\textcolor[rgb]{0.13,0.29,0.53}{\textbf{#1}}}
\newcommand{\DataTypeTok}[1]{\textcolor[rgb]{0.13,0.29,0.53}{#1}}
\newcommand{\DecValTok}[1]{\textcolor[rgb]{0.00,0.00,0.81}{#1}}
\newcommand{\DocumentationTok}[1]{\textcolor[rgb]{0.56,0.35,0.01}{\textbf{\textit{#1}}}}
\newcommand{\ErrorTok}[1]{\textcolor[rgb]{0.64,0.00,0.00}{\textbf{#1}}}
\newcommand{\ExtensionTok}[1]{#1}
\newcommand{\FloatTok}[1]{\textcolor[rgb]{0.00,0.00,0.81}{#1}}
\newcommand{\FunctionTok}[1]{\textcolor[rgb]{0.00,0.00,0.00}{#1}}
\newcommand{\ImportTok}[1]{#1}
\newcommand{\InformationTok}[1]{\textcolor[rgb]{0.56,0.35,0.01}{\textbf{\textit{#1}}}}
\newcommand{\KeywordTok}[1]{\textcolor[rgb]{0.13,0.29,0.53}{\textbf{#1}}}
\newcommand{\NormalTok}[1]{#1}
\newcommand{\OperatorTok}[1]{\textcolor[rgb]{0.81,0.36,0.00}{\textbf{#1}}}
\newcommand{\OtherTok}[1]{\textcolor[rgb]{0.56,0.35,0.01}{#1}}
\newcommand{\PreprocessorTok}[1]{\textcolor[rgb]{0.56,0.35,0.01}{\textit{#1}}}
\newcommand{\RegionMarkerTok}[1]{#1}
\newcommand{\SpecialCharTok}[1]{\textcolor[rgb]{0.00,0.00,0.00}{#1}}
\newcommand{\SpecialStringTok}[1]{\textcolor[rgb]{0.31,0.60,0.02}{#1}}
\newcommand{\StringTok}[1]{\textcolor[rgb]{0.31,0.60,0.02}{#1}}
\newcommand{\VariableTok}[1]{\textcolor[rgb]{0.00,0.00,0.00}{#1}}
\newcommand{\VerbatimStringTok}[1]{\textcolor[rgb]{0.31,0.60,0.02}{#1}}
\newcommand{\WarningTok}[1]{\textcolor[rgb]{0.56,0.35,0.01}{\textbf{\textit{#1}}}}
\usepackage{longtable,booktabs}
% Fix footnotes in tables (requires footnote package)
\IfFileExists{footnote.sty}{\usepackage{footnote}\makesavenoteenv{longtable}}{}
\usepackage{graphicx,grffile}
\makeatletter
\def\maxwidth{\ifdim\Gin@nat@width>\linewidth\linewidth\else\Gin@nat@width\fi}
\def\maxheight{\ifdim\Gin@nat@height>\textheight\textheight\else\Gin@nat@height\fi}
\makeatother
% Scale images if necessary, so that they will not overflow the page
% margins by default, and it is still possible to overwrite the defaults
% using explicit options in \includegraphics[width, height, ...]{}
\setkeys{Gin}{width=\maxwidth,height=\maxheight,keepaspectratio}
\setlength{\emergencystretch}{3em}  % prevent overfull lines
\providecommand{\tightlist}{%
  \setlength{\itemsep}{0pt}\setlength{\parskip}{0pt}}
\setcounter{secnumdepth}{0}
% Redefines (sub)paragraphs to behave more like sections
\ifx\paragraph\undefined\else
\let\oldparagraph\paragraph
\renewcommand{\paragraph}[1]{\oldparagraph{#1}\mbox{}}
\fi
\ifx\subparagraph\undefined\else
\let\oldsubparagraph\subparagraph
\renewcommand{\subparagraph}[1]{\oldsubparagraph{#1}\mbox{}}
\fi

% set default figure placement to htbp
\makeatletter
\def\fps@figure{htbp}
\makeatother


\title{Practica2}
\author{Maite Piedra}
\date{3/6/2020}

\begin{document}
\maketitle

\begin{Shaded}
\begin{Highlighting}[]
\CommentTok{#install.packages("knitr")}
\CommentTok{#install.packages("VIM")}
\CommentTok{#install.packages("caret")}
\CommentTok{#install.packages("ggplot2")}
\CommentTok{#install.packages("corrplot")}
\KeywordTok{library}\NormalTok{(caret)}
\end{Highlighting}
\end{Shaded}

\begin{verbatim}
## Loading required package: lattice
\end{verbatim}

\begin{verbatim}
## Loading required package: ggplot2
\end{verbatim}

\begin{Shaded}
\begin{Highlighting}[]
\KeywordTok{library}\NormalTok{(knitr)}
\KeywordTok{library}\NormalTok{(VIM)}
\end{Highlighting}
\end{Shaded}

\begin{verbatim}
## Loading required package: colorspace
\end{verbatim}

\begin{verbatim}
## Loading required package: grid
\end{verbatim}

\begin{verbatim}
## VIM is ready to use.
\end{verbatim}

\begin{verbatim}
## Suggestions and bug-reports can be submitted at: https://github.com/statistikat/VIM/issues
\end{verbatim}

\begin{verbatim}
## 
## Attaching package: 'VIM'
\end{verbatim}

\begin{verbatim}
## The following object is masked from 'package:datasets':
## 
##     sleep
\end{verbatim}

\begin{Shaded}
\begin{Highlighting}[]
\KeywordTok{library}\NormalTok{(ggplot2)}
\KeywordTok{library}\NormalTok{(corrplot)}
\end{Highlighting}
\end{Shaded}

\begin{verbatim}
## corrplot 0.84 loaded
\end{verbatim}

\begin{Shaded}
\begin{Highlighting}[]
\KeywordTok{setwd}\NormalTok{(}\StringTok{'/Users/maitepiedrayera/Developer/uoc/Tipologia_ciclo_vida_datos/data/'}\NormalTok{)}
\NormalTok{ruta =}\StringTok{ }\KeywordTok{paste}\NormalTok{(}\KeywordTok{getwd}\NormalTok{(),}\StringTok{"hotel_bookings.csv"}\NormalTok{, }\DataTypeTok{sep =} \StringTok{"/"}\NormalTok{)}
\NormalTok{dh <-}\StringTok{ }\KeywordTok{read.csv}\NormalTok{(ruta, }\DataTypeTok{header =} \OtherTok{TRUE}\NormalTok{, }\DataTypeTok{encoding =} \StringTok{"latin1"}\NormalTok{ ,}\DataTypeTok{sep =} \StringTok{","}\NormalTok{, }\DataTypeTok{stringsAsFactors =} \OtherTok{TRUE}\NormalTok{)}
\end{Highlighting}
\end{Shaded}

\hypertarget{descripciuxf3n-del-dataset.-por-quuxe9-es-importante-y-quuxe9-preguntaproblema-pretende-responder}{%
\subsection{1.- Descripción del dataset. ¿Por qué es importante y qué
pregunta/problema pretende
responder?}\label{descripciuxf3n-del-dataset.-por-quuxe9-es-importante-y-quuxe9-preguntaproblema-pretende-responder}}

Como podemos ver nuestro dataset es bastante grande, esta formado por
119390 filas y 32 columnas

\begin{Shaded}
\begin{Highlighting}[]
\KeywordTok{str}\NormalTok{(dh)}
\end{Highlighting}
\end{Shaded}

\begin{verbatim}
## 'data.frame':    119390 obs. of  32 variables:
##  $ hotel                         : Factor w/ 2 levels "City Hotel","Resort Hotel": 2 2 2 2 2 2 2 2 2 2 ...
##  $ is_canceled                   : int  0 0 0 0 0 0 0 0 1 1 ...
##  $ lead_time                     : int  342 737 7 13 14 14 0 9 85 75 ...
##  $ arrival_date_year             : int  2015 2015 2015 2015 2015 2015 2015 2015 2015 2015 ...
##  $ arrival_date_month            : Factor w/ 12 levels "April","August",..: 6 6 6 6 6 6 6 6 6 6 ...
##  $ arrival_date_week_number      : int  27 27 27 27 27 27 27 27 27 27 ...
##  $ arrival_date_day_of_month     : int  1 1 1 1 1 1 1 1 1 1 ...
##  $ stays_in_weekend_nights       : int  0 0 0 0 0 0 0 0 0 0 ...
##  $ stays_in_week_nights          : int  0 0 1 1 2 2 2 2 3 3 ...
##  $ adults                        : int  2 2 1 1 2 2 2 2 2 2 ...
##  $ children                      : int  0 0 0 0 0 0 0 0 0 0 ...
##  $ babies                        : int  0 0 0 0 0 0 0 0 0 0 ...
##  $ meal                          : Factor w/ 5 levels "BB","FB","HB",..: 1 1 1 1 1 1 1 2 1 3 ...
##  $ country                       : Factor w/ 178 levels "ABW","AGO","AIA",..: 137 137 60 60 60 60 137 137 137 137 ...
##  $ market_segment                : Factor w/ 8 levels "Aviation","Complementary",..: 4 4 4 3 7 7 4 4 7 6 ...
##  $ distribution_channel          : Factor w/ 5 levels "Corporate","Direct",..: 2 2 2 1 4 4 2 2 4 4 ...
##  $ is_repeated_guest             : int  0 0 0 0 0 0 0 0 0 0 ...
##  $ previous_cancellations        : int  0 0 0 0 0 0 0 0 0 0 ...
##  $ previous_bookings_not_canceled: int  0 0 0 0 0 0 0 0 0 0 ...
##  $ reserved_room_type            : Factor w/ 10 levels "A","B","C","D",..: 3 3 1 1 1 1 3 3 1 4 ...
##  $ assigned_room_type            : Factor w/ 12 levels "A","B","C","D",..: 3 3 3 1 1 1 3 3 1 4 ...
##  $ booking_changes               : int  3 4 0 0 0 0 0 0 0 0 ...
##  $ deposit_type                  : Factor w/ 3 levels "No Deposit","Non Refund",..: 1 1 1 1 1 1 1 1 1 1 ...
##  $ agent                         : Factor w/ 334 levels "1","10","103",..: 334 334 334 157 103 103 334 156 103 40 ...
##  $ company                       : Factor w/ 353 levels "10","100","101",..: 353 353 353 353 353 353 353 353 353 353 ...
##  $ days_in_waiting_list          : int  0 0 0 0 0 0 0 0 0 0 ...
##  $ customer_type                 : Factor w/ 4 levels "Contract","Group",..: 3 3 3 3 3 3 3 3 3 3 ...
##  $ adr                           : num  0 0 75 75 98 ...
##  $ required_car_parking_spaces   : int  0 0 0 0 0 0 0 0 0 0 ...
##  $ total_of_special_requests     : int  0 0 0 0 1 1 0 1 1 0 ...
##  $ reservation_status            : Factor w/ 3 levels "Canceled","Check-Out",..: 2 2 2 2 2 2 2 2 1 1 ...
##  $ reservation_status_date       : Factor w/ 926 levels "2014-10-17","2014-11-18",..: 122 122 123 123 124 124 124 124 73 62 ...
\end{verbatim}

A continuacion describimos cada una de las columnas que conforman
nuestro dataset

1.- Hotel :Hotel (H1 = Resort Hotel or H2 = City Hotel)

2.- Is\_canceled: Indica si la reserva fue cancelada (1) o no (0)

3.- Lead\_time: Número de días transcurridos entre la fecha de entrada
de la reserva en sistema de gestión de reservas (PMS) y la fecha de
llegada

4.- Arrival\_date\_year: Año

5.- Arrival\_date\_month: Mes

6.- Arrival\_date\_week\_number: Día de la semana

7.- Arrival\_date\_day\_of\_month: Día del mes en el que entró en el
hotel

8.- Stays\_in\_weekend\_nights: Número de noches de fin de semana
(sábado o domingo) que el huésped se hospedó o reservó para quedarse en
el hotel

9.- Stays\_in\_week\_nights: Número de noches entre semana (Lunes a
viernes) que el huésped se hospedó o reservó para quedarse en el hotel

10.- Adults: Número de adultos.

11.- Children: Número de niños.

12.- Babies: Número de bebes.

13.- Meal: Type of meal booked: Las categorías se presentan en paquetes
estándar de comidas de hospitalidad: Indefinido / SC - sin paquete de
comidas; BB - Alojamiento y desayuno; HB - Media pensión (desayuno y
otra comida, generalmente cena); FB - Pensión completa (desayuno,
almuerzo y cena)

14.- Country: País de origen.

15.- Market\_segment: Designación del segmento de mercado. En
categorías, el término ``TA'' significa ``Agentes de viajes'' y ``TO''
significa ``Operadores turísticos''.

16.- Distribution\_channel: Canal de distribución de reservas. El
término ``TA'' significa ``Agentes de viajes'' y ``TO'' significa
``Operadores turísticos''.

17.-Is\_repeated\_guest:Valor que indica si el nombre de la reserva era
de un huésped repetido (1) o no (0).

18.- Previous\_cancellations: Número de reservas anteriores que el
cliente canceló antes de la reserva actual.

19.- Previous\_bookings\_not\_canceled: Número de reservas anteriores no
canceladas por el cliente antes de la reserva actual.

20.- Reserved\_room\_type: Código de tipo de habitación reservado. El
código se presenta en lugar de la designación por razones de anonimato.

21.- Assigned\_room\_type: Código para el tipo de habitación asignada a
la reserva. A veces, el tipo de habitación asignada difiere del tipo de
habitación reservada debido a razones de operación del hotel (por
ejemplo, sobreventa) o por solicitud del cliente. El código se presenta
en lugar de la designación por razones de anonimato.

22.- Booking\_changes:Número de cambios / modificaciones realizados en
la reserva desde el momento en que se ingresó en el PMS (sistema de
gestión de reservas) hasta el momento del check-in o cancelación.

23.- Deposit\_type: Indicación de si el cliente realizó un depósito para
garantizar la reserva. Esta variable puede asumir tres categorías: Sin
depósito: no se realizó ningún depósito; Sin reembolso: se realizó un
depósito por el valor del costo total de la estadía; Reembolsable: se
realizó un depósito con un valor por debajo del costo total de la
estadía.

24.- Agent: Identificación de la agencia de viajes que realizó la
reserva.

25.-Company: Identificación de la empresa / entidad que realizó la
reserva o responsable de pagar la reserva. Se presenta la identificación
en lugar de la designación por razones de anonimato.

26.- Days\_in\_waiting\_list: Número de días que la reserva estuvo en la
lista de espera antes de ser confirmada al cliente.

27.- Customer\_type: Tipo de reserva, asumiendo una de cuatro
categorías: Contrato: cuando la reserva tiene una asignación u otro tipo
de contrato asociado; Grupo: cuando la reserva está asociada a un grupo;
Transitoria: cuando la reserva no forma parte de un grupo o contrato, y
no está asociada a otra reserva transitoria; Parte transitoria: cuando
la reserva es transitoria, pero está asociada a al menos otra reserva
transitoria

28.- Adr: Tarifa diaria promedio según se define dividiendo la suma de
todas las transacciones de alojamiento por el número total de noches de
estadía.

29.- Required\_car\_parking\_spaces: Número de plazas de aparcamiento
requeridas por el cliente.

30-.Total\_of\_special\_requests: Número de solicitudes especiales
realizadas por el cliente (por ejemplo, cama doble o piso alto).

31.- Reservation\_status: Último estado de la reserva, asumiendo una de
tres categorías: Cancelada: la reserva fue cancelada por el cliente;
check-out: el cliente se ha registrado pero ya se ha ido; No-Show: el
cliente no hizo el check-in e informó al hotel del motivo.

32.- Reservation\_status\_date: Fecha en la que se estableció el último
estado. Esta variable se puede usar junto con el Estado de reserva para
comprender cuándo se canceló la reserva o cuándo el cliente realizó el
check-out del hotel

Con este dataset, responder a preguntas como: ¿ Cual es el mejor epoca
del año para hacer una reserva? ¿ Cual es el número de dias optimo para
que la estadia en un hotel me salga rentable?

\hypertarget{integraciuxf3n-y-selecciuxf3n-de-los-datos-de-interuxe9s-a-analizar.}{%
\subsection{2.- Integración y selección de los datos de interés a
analizar.}\label{integraciuxf3n-y-selecciuxf3n-de-los-datos-de-interuxe9s-a-analizar.}}

A partir de todas las variables que tenemos se nos plantea cuales son
las mas indicadas para resolver nuestras preguntas, como tenemos muchas
variables, vamos a eliminar aquellas que creemos que no tienen especial
relevancia para las preguntas que nos ocupan, y luego procederemos a
limpiar el dataset.

En nuestro caso nos quedamos con 13 variables que consideramos
importantes para nuestro estudio y eliminamos del dataset las
siguientes: Is\_canceled, Lead\_time, Arrival\_date\_year, meal,
Distribution\_channel, Previous\_cancellations,
Previous\_bookings\_not\_canceled, Reserved\_room\_type,
Assigned\_room\_type, Booking\_changes, Agent, Company,
Days\_in\_waiting\_list, Customer\_type, Required\_car\_parking\_spaces,
Total\_of\_special\_requests, Reservation\_status,
Reservation\_status\_date, country y market\_segment.

\begin{Shaded}
\begin{Highlighting}[]
\CommentTok{#Nos quedamos solo con las reservas que no han sido cancela, osea aquella cuyo valor is_canceled = 0}
\CommentTok{# Asi que pasamos de 119390 filas a 75166 filas}
\NormalTok{dh_reducido =}\StringTok{ }\NormalTok{(dh[dh}\OperatorTok{$}\NormalTok{is_canceled }\OperatorTok{==}\StringTok{ }\DecValTok{0}\NormalTok{, ])}

\CommentTok{# eliminamos las siguientes columnas}
\NormalTok{dh_reducido =}\StringTok{ }\NormalTok{dh_reducido[, }\OperatorTok{-}\KeywordTok{c}\NormalTok{(}\DecValTok{2}\OperatorTok{:}\DecValTok{4}\NormalTok{,}\DecValTok{13}\OperatorTok{:}\DecValTok{16}\NormalTok{,}\DecValTok{18}\OperatorTok{:}\DecValTok{22}\NormalTok{,}\DecValTok{24}\OperatorTok{:}\DecValTok{26}\NormalTok{,}\DecValTok{29}\OperatorTok{:}\DecValTok{32}\NormalTok{ )]}

\CommentTok{# vemos de que tipo es cada variable en nuestro dataset}
\NormalTok{tipo_dato =}\StringTok{ }\KeywordTok{sapply}\NormalTok{(dh_reducido, }\ControlFlowTok{function}\NormalTok{(x) }\KeywordTok{class}\NormalTok{(x))}
\KeywordTok{kable}\NormalTok{(}\KeywordTok{data.frame}\NormalTok{(}\DataTypeTok{variables =} \KeywordTok{names}\NormalTok{(dh_reducido), }\DataTypeTok{tipo_variable =} \KeywordTok{as.vector}\NormalTok{(tipo_dato)))}
\end{Highlighting}
\end{Shaded}

\begin{longtable}[]{@{}ll@{}}
\toprule
variables & tipo\_variable\tabularnewline
\midrule
\endhead
hotel & factor\tabularnewline
arrival\_date\_month & factor\tabularnewline
arrival\_date\_week\_number & integer\tabularnewline
arrival\_date\_day\_of\_month & integer\tabularnewline
stays\_in\_weekend\_nights & integer\tabularnewline
stays\_in\_week\_nights & integer\tabularnewline
adults & integer\tabularnewline
children & integer\tabularnewline
babies & integer\tabularnewline
is\_repeated\_guest & integer\tabularnewline
deposit\_type & factor\tabularnewline
customer\_type & factor\tabularnewline
adr & numeric\tabularnewline
\bottomrule
\end{longtable}

\begin{Shaded}
\begin{Highlighting}[]
\CommentTok{# convertimos a numeric aquellos valores que son enteros, para poder trabajar mas facilmente con ellos.}
\NormalTok{dh_reducido[, }\KeywordTok{c}\NormalTok{(}\DecValTok{3}\OperatorTok{:}\DecValTok{10}\NormalTok{)] <-}\StringTok{ }\KeywordTok{sapply}\NormalTok{(dh_reducido[, }\KeywordTok{c}\NormalTok{(}\DecValTok{3}\OperatorTok{:}\DecValTok{10}\NormalTok{)], as.numeric)}
\NormalTok{tipo_dato_reducido =}\StringTok{ }\KeywordTok{sapply}\NormalTok{(dh_reducido, }\ControlFlowTok{function}\NormalTok{(x) }\KeywordTok{class}\NormalTok{(x))}
\KeywordTok{kable}\NormalTok{(}\KeywordTok{data.frame}\NormalTok{(}\DataTypeTok{variables =} \KeywordTok{names}\NormalTok{(dh_reducido), }\DataTypeTok{tipo_variable =} \KeywordTok{as.vector}\NormalTok{(tipo_dato_reducido)))}
\end{Highlighting}
\end{Shaded}

\begin{longtable}[]{@{}ll@{}}
\toprule
variables & tipo\_variable\tabularnewline
\midrule
\endhead
hotel & factor\tabularnewline
arrival\_date\_month & factor\tabularnewline
arrival\_date\_week\_number & numeric\tabularnewline
arrival\_date\_day\_of\_month & numeric\tabularnewline
stays\_in\_weekend\_nights & numeric\tabularnewline
stays\_in\_week\_nights & numeric\tabularnewline
adults & numeric\tabularnewline
children & numeric\tabularnewline
babies & numeric\tabularnewline
is\_repeated\_guest & numeric\tabularnewline
deposit\_type & factor\tabularnewline
customer\_type & factor\tabularnewline
adr & numeric\tabularnewline
\bottomrule
\end{longtable}

\begin{Shaded}
\begin{Highlighting}[]
\CommentTok{# el resumende nuestro dataset seria el siguiente }
\KeywordTok{summary}\NormalTok{(dh_reducido)}
\end{Highlighting}
\end{Shaded}

\begin{verbatim}
##           hotel       arrival_date_month arrival_date_week_number
##  City Hotel  :46228   August : 8638      Min.   : 1.00           
##  Resort Hotel:28938   July   : 7919      1st Qu.:16.00           
##                       May    : 7114      Median :28.00           
##                       October: 6914      Mean   :27.08           
##                       March  : 6645      3rd Qu.:38.00           
##                       April  : 6565      Max.   :53.00           
##                       (Other):31371                              
##  arrival_date_day_of_month stays_in_weekend_nights stays_in_week_nights
##  Min.   : 1.00             Min.   : 0.000          Min.   : 0.000      
##  1st Qu.: 8.00             1st Qu.: 0.000          1st Qu.: 1.000      
##  Median :16.00             Median : 1.000          Median : 2.000      
##  Mean   :15.84             Mean   : 0.929          Mean   : 2.464      
##  3rd Qu.:23.00             3rd Qu.: 2.000          3rd Qu.: 3.000      
##  Max.   :31.00             Max.   :19.000          Max.   :50.000      
##                                                                        
##      adults        children          babies         is_repeated_guest
##  Min.   :0.00   Min.   :0.0000   Min.   : 0.00000   Min.   :0.00000  
##  1st Qu.:2.00   1st Qu.:0.0000   1st Qu.: 0.00000   1st Qu.:0.00000  
##  Median :2.00   Median :0.0000   Median : 0.00000   Median :0.00000  
##  Mean   :1.83   Mean   :0.1023   Mean   : 0.01038   Mean   :0.04334  
##  3rd Qu.:2.00   3rd Qu.:0.0000   3rd Qu.: 0.00000   3rd Qu.:0.00000  
##  Max.   :4.00   Max.   :3.0000   Max.   :10.00000   Max.   :1.00000  
##                                                                      
##      deposit_type           customer_type        adr        
##  No Deposit:74947   Contract       : 2814   Min.   : -6.38  
##  Non Refund:   93   Group          :  518   1st Qu.: 67.50  
##  Refundable:  126   Transient      :53099   Median : 92.50  
##                     Transient-Party:18735   Mean   : 99.99  
##                                             3rd Qu.:125.00  
##                                             Max.   :510.00  
## 
\end{verbatim}

\hypertarget{limpieza-de-los-datos.}{%
\subsection{3.- Limpieza de los datos.}\label{limpieza-de-los-datos.}}

Analizaremos uno por uno los datos devueltos por nuestro summary -
Hotel: los valores que toma la variable con City Hotel y Resort Hotel,
osea son los valores esperados. - is\_canceled: vemos que el min valor
es 0 y el max 1 y esa variable se mueve entre esos valores, odea esta
correcto. - arrival\_date\_month: este valor prodriamos normalizarlo,
asignando un numero a cada mes (1 =\textgreater{} Enero..
12=\textgreater{} Diciembre), asi nos aseguramos que los valores se
mueven entre 1 y 12 y con eso eliminamos la posibilidad de que haya un
mes mal escrito o algo por el estilo. - arrival\_date\_week\_number :
tenemos que el menor valor es 1 (llegar en la primera semana) y el mayor
valor 53, sabiendo que un año tiene 53 semanas lo podemos considerar
correcto. - arrival\_date\_day\_of\_month: En principio estaria bien ,
ya que se mueve entre 1 y 31, osea los dias de los meses. -
stays\_in\_weekend\_nights: numero de noches que caen en fin de semana,
tenemos un valor sospechoso, que puede ser un posible outlier, pasar 19
fines de semana en un hotel es bastante raro. - stays\_in\_week\_nights:
pasa lo mismo que en el caso anterior, seria una variable a examinar un
poco mas a fondo, ya que tenemos un valor bastente inusual (50). -
adults: tenemos un valor un poco extraño igual que en los casos
anteriores que es un posible outlier, 55 adultos en una reserva, en este
caso cabe la posibilidad que se auna reserva de grupo, pero igualmente
es algo que hay que mirar. - children: Lo mismo con esta variable, es
raro una reserva donde hayan 10 niños, aunque puede ser de una reserva
de grupo, y ademas vemos que tenemos valores perdidos\\
- babies: pasa lo mismo que con children.\\
- market\_segment - is\_repeated\_guest : En principio estaria correcto,
toma valores entre 0 y 1. - deposit\_type: estaria correcto, pues solo
hay tres tipos de deposito. - customer\_type: Estaría correcto, pues
tenemos 4 tipos de cliente - adr: este valor equivale a la tarifa diaria
promedio, y es una variable que tenemos que mirar pues toma valores
negativos y valores 0 cosa que no deberia ser ademas su valor mas alto
es de 5400, que puede ser un posible outlier.

Una vez analizadas las variables una por una, vamos a realizar los
cambios que dejimos anteriormente.

\begin{Shaded}
\begin{Highlighting}[]
\NormalTok{dh_reducido}\OperatorTok{$}\NormalTok{arrival_date_month <-}\StringTok{ }\KeywordTok{factor}\NormalTok{(dh_reducido}\OperatorTok{$}\NormalTok{arrival_date_month,}
                  \DataTypeTok{levels =} \KeywordTok{c}\NormalTok{(}\StringTok{"January"}\NormalTok{,}\StringTok{"February"}\NormalTok{,}\StringTok{"March"}\NormalTok{,}\StringTok{"April"}\NormalTok{,}\StringTok{"May"}\NormalTok{,}\StringTok{"June"}\NormalTok{, }\StringTok{"July"}\NormalTok{,}\StringTok{"August"}\NormalTok{, }\StringTok{"September"}\NormalTok{, }\StringTok{"October"}\NormalTok{, }\StringTok{"November"}\NormalTok{, }\StringTok{"December"}\NormalTok{),}
                  \DataTypeTok{labels =} \KeywordTok{c}\NormalTok{(}\DecValTok{1}\NormalTok{,}\DecValTok{2}\NormalTok{,}\DecValTok{3}\NormalTok{,}\DecValTok{4}\NormalTok{,}\DecValTok{5}\NormalTok{,}\DecValTok{6}\NormalTok{,}\DecValTok{7}\NormalTok{,}\DecValTok{8}\NormalTok{,}\DecValTok{9}\NormalTok{,}\DecValTok{10}\NormalTok{,}\DecValTok{11}\NormalTok{,}\DecValTok{12}\NormalTok{))}

\NormalTok{dh_reducido}\OperatorTok{$}\NormalTok{arrival_date_month =}\StringTok{ }\KeywordTok{as.numeric}\NormalTok{(}\KeywordTok{as.character}\NormalTok{(dh_reducido}\OperatorTok{$}\NormalTok{arrival_date_month))}
\end{Highlighting}
\end{Shaded}

\hypertarget{los-datos-contienen-ceros-o-elementos-vacuxedos-cuxf3mo-gestionaruxedas-cada-uno-de-estos-casos}{%
\paragraph{3.1. ¿Los datos contienen ceros o elementos vacíos? ¿Cómo
gestionarías cada uno de estos
casos?}\label{los-datos-contienen-ceros-o-elementos-vacuxedos-cuxf3mo-gestionaruxedas-cada-uno-de-estos-casos}}

Es normal que al trabajar con un gran volumen de datos tengamos datos
que esten desaparecidos o incluso que tengan valores erroneos.

Los valores 0 o vacios (NA) pueden aparecer porque se desconoce el dato
y se dejo vacio en su momento, o porque la persona o algoritmos que
tenia que introducir los datos en ese momento tuvo algun problema, en el
caso de una persona fisica, se puede deber a un olvido, y en el caso de
un programa/algoritmo un fallo en el método de recoleccion de datos o de
escritura.

\begin{itemize}
\item
  Rellenar los valores faltantes con el valor de la media, de todos los
  valores obtenidos, no es una técnica muy optima pero no eliminamos las
  filas.
\item
  Rellenar los valores faltantes usando el algorimo Knn, osea usa a los
  vecinos mas cercanos para predecir que valor debería ser ese valor
  faltante.
\end{itemize}

en nuestro caso teniamos valores faltantes antes de reducir los datos en
particular 4 en la variable children, al eliminar de los datos las
reservas canceladas estos desaparecieron.

\hypertarget{identificaciuxf3n-y-tratamiento-de-valores-extremos.}{%
\paragraph{3.2. Identificación y tratamiento de valores
extremos.}\label{identificaciuxf3n-y-tratamiento-de-valores-extremos.}}

Vamos a explicar como podemos detectar los valores extremos a parte e
graficamente con el boxplot como lo podemos saber numericamente. Para
ello usamos la funcion summary que nos muestra los quartiles de los
variables.

Sabemos que entre Q1 y Q3 se encuentra el 50\% de los valores obtenidos
en el estudio y esta distancia de llama distancia Intercuartilica (IQR).

Un valor atípico leve se define como aquel que esta 1,5 veces el rango
intercuartilico por debajo de Q1 o por encima de Q3 Y un valor atípico
extremo se define como aquel que esta 3 veces el rango intercuartilico
por debajo de Q1 o por encima de Q3

Cuando la mediana esta muy distante de la media (casi el doble) podemos
decir que pasa algo raro, osea que hay valores tan altos que estan
tragiversando el estudio.

Ahora aplicamos esto para cada caso.

\begin{Shaded}
\begin{Highlighting}[]
\KeywordTok{attach}\NormalTok{(dh_reducido)}

\CommentTok{#Crearemos una funcion que nos haga los calculos para no repetirlos con cada caso.}
\NormalTok{outlierReplace =}\StringTok{ }\ControlFlowTok{function}\NormalTok{(entrada)\{}
\NormalTok{  iqr.valor =}\StringTok{ }\KeywordTok{IQR}\NormalTok{(entrada)}
\NormalTok{  cuantiles =}\StringTok{ }\KeywordTok{quantile}\NormalTok{(entrada, }\KeywordTok{c}\NormalTok{(}\FloatTok{0.25}\NormalTok{, }\FloatTok{0.50}\NormalTok{, }\FloatTok{0.75}\NormalTok{))}
  \CommentTok{# Todo valor inferior a este se considera oulier}
\NormalTok{  outlier_min =}\StringTok{ }\KeywordTok{as.numeric}\NormalTok{(cuantiles[}\DecValTok{1}\NormalTok{])}\OperatorTok{-}\FloatTok{1.5}\OperatorTok{*}\NormalTok{iqr.valor }
  \CommentTok{# Todo valor superior a este se considera oulier}
\NormalTok{  outlier_max =}\StringTok{ }\KeywordTok{as.numeric}\NormalTok{(cuantiles[}\DecValTok{3}\NormalTok{])}\OperatorTok{+}\FloatTok{1.5}\OperatorTok{*}\NormalTok{iqr.valor}
  \CommentTok{#ahora reemplazamos el dato por la media}
\NormalTok{  entrada[entrada }\OperatorTok{<}\StringTok{ }\NormalTok{outlier_min] =}\StringTok{ }\KeywordTok{round}\NormalTok{(}\KeywordTok{mean}\NormalTok{(entrada))}
\NormalTok{  entrada[entrada }\OperatorTok{>}\StringTok{ }\NormalTok{outlier_max] =}\StringTok{ }\KeywordTok{round}\NormalTok{(}\KeywordTok{median}\NormalTok{(entrada))}
  \KeywordTok{return}\NormalTok{(entrada)}
\NormalTok{\} }

\KeywordTok{boxplot}\NormalTok{(dh_reducido}\OperatorTok{$}\NormalTok{stays_in_weekend_nights, }\DataTypeTok{main=}\StringTok{"Estadía de dias los fines de semana"}\NormalTok{)}
\end{Highlighting}
\end{Shaded}

\includegraphics{prac2_files/figure-latex/valores_extremos-1.pdf}

\begin{Shaded}
\begin{Highlighting}[]
\KeywordTok{summary}\NormalTok{(dh_reducido}\OperatorTok{$}\NormalTok{stays_in_weekend_nights)}
\end{Highlighting}
\end{Shaded}

\begin{verbatim}
##    Min. 1st Qu.  Median    Mean 3rd Qu.    Max. 
##   0.000   0.000   1.000   0.929   2.000  19.000
\end{verbatim}

\begin{Shaded}
\begin{Highlighting}[]
\KeywordTok{boxplot}\NormalTok{(dh_reducido}\OperatorTok{$}\NormalTok{stays_in_week_nights, }\DataTypeTok{main=}\StringTok{"Estadía de dias los fines de semana"}\NormalTok{)}
\end{Highlighting}
\end{Shaded}

\includegraphics{prac2_files/figure-latex/valores_extremos-2.pdf}

\begin{Shaded}
\begin{Highlighting}[]
\KeywordTok{summary}\NormalTok{(dh_reducido}\OperatorTok{$}\NormalTok{stays_in_week_nights)}
\end{Highlighting}
\end{Shaded}

\begin{verbatim}
##    Min. 1st Qu.  Median    Mean 3rd Qu.    Max. 
##   0.000   1.000   2.000   2.464   3.000  50.000
\end{verbatim}

\begin{Shaded}
\begin{Highlighting}[]
\KeywordTok{boxplot}\NormalTok{(dh_reducido}\OperatorTok{$}\NormalTok{adults, }\DataTypeTok{main=}\StringTok{"Adultos"}\NormalTok{)}
\end{Highlighting}
\end{Shaded}

\includegraphics{prac2_files/figure-latex/valores_extremos-3.pdf}

\begin{Shaded}
\begin{Highlighting}[]
\KeywordTok{summary}\NormalTok{(dh_reducido}\OperatorTok{$}\NormalTok{adults)}
\end{Highlighting}
\end{Shaded}

\begin{verbatim}
##    Min. 1st Qu.  Median    Mean 3rd Qu.    Max. 
##    0.00    2.00    2.00    1.83    2.00    4.00
\end{verbatim}

\begin{Shaded}
\begin{Highlighting}[]
\KeywordTok{boxplot}\NormalTok{(dh_reducido}\OperatorTok{$}\NormalTok{children, }\DataTypeTok{main=}\StringTok{"Niños"}\NormalTok{)}
\end{Highlighting}
\end{Shaded}

\includegraphics{prac2_files/figure-latex/valores_extremos-4.pdf}

\begin{Shaded}
\begin{Highlighting}[]
\KeywordTok{summary}\NormalTok{(dh_reducido}\OperatorTok{$}\NormalTok{children)}
\end{Highlighting}
\end{Shaded}

\begin{verbatim}
##    Min. 1st Qu.  Median    Mean 3rd Qu.    Max. 
##  0.0000  0.0000  0.0000  0.1023  0.0000  3.0000
\end{verbatim}

\begin{Shaded}
\begin{Highlighting}[]
\KeywordTok{boxplot}\NormalTok{(dh_reducido}\OperatorTok{$}\NormalTok{babies, }\DataTypeTok{main=}\StringTok{"Bebes"}\NormalTok{)}
\end{Highlighting}
\end{Shaded}

\includegraphics{prac2_files/figure-latex/valores_extremos-5.pdf}

\begin{Shaded}
\begin{Highlighting}[]
\KeywordTok{summary}\NormalTok{(dh_reducido}\OperatorTok{$}\NormalTok{babies)}
\end{Highlighting}
\end{Shaded}

\begin{verbatim}
##     Min.  1st Qu.   Median     Mean  3rd Qu.     Max. 
##  0.00000  0.00000  0.00000  0.01038  0.00000 10.00000
\end{verbatim}

\begin{Shaded}
\begin{Highlighting}[]
\KeywordTok{boxplot}\NormalTok{(dh_reducido}\OperatorTok{$}\NormalTok{adr, }\DataTypeTok{main=}\StringTok{"tarifa diaria promedio"}\NormalTok{)}
\end{Highlighting}
\end{Shaded}

\includegraphics{prac2_files/figure-latex/valores_extremos-6.pdf}

\begin{Shaded}
\begin{Highlighting}[]
\KeywordTok{summary}\NormalTok{(dh_reducido}\OperatorTok{$}\NormalTok{adr)}
\end{Highlighting}
\end{Shaded}

\begin{verbatim}
##    Min. 1st Qu.  Median    Mean 3rd Qu.    Max. 
##   -6.38   67.50   92.50   99.99  125.00  510.00
\end{verbatim}

En nuestro caso a pesar de haber creado una funcion para eliminar los
valores outlier, no la hemos necesitado, puesto que tenemos valores
outlier pero como bien dice la teoria la media y la mediana no distan
mucho entre si, como vemos en el summary de cada variable, por lo tanto
no consideramos los valores outlier peligroso o dañinos para nuestro
estudio, es decir si los corregimos el resultado final con los valores
outlier o sin ellos no será muy diferente. Por lo tanto los dejamos.

\begin{Shaded}
\begin{Highlighting}[]
\CommentTok{# Extraemos en un CSV los datos finales que usaremos.}
\KeywordTok{write.csv}\NormalTok{(dh_reducido, }\StringTok{"datos_hoteles.csv"}\NormalTok{)}
\end{Highlighting}
\end{Shaded}

\hypertarget{anuxe1lisis-de-los-datos.}{%
\subsection{4.- Análisis de los
datos.}\label{anuxe1lisis-de-los-datos.}}

\hypertarget{selecciuxf3n-de-los-grupos-de-datos-que-se-quieren-analizarcomparar-planificaciuxf3n-de-los-anuxe1lisis-a-aplicar.}{%
\paragraph{4.1. Selección de los grupos de datos que se quieren
analizar/comparar (planificación de los análisis a
aplicar).}\label{selecciuxf3n-de-los-grupos-de-datos-que-se-quieren-analizarcomparar-planificaciuxf3n-de-los-anuxe1lisis-a-aplicar.}}

\hypertarget{comprobaciuxf3n-de-la-normalidad-y-homogeneidad-de-la-varianza.}{%
\paragraph{4.2. Comprobación de la normalidad y homogeneidad de la
varianza.}\label{comprobaciuxf3n-de-la-normalidad-y-homogeneidad-de-la-varianza.}}

\begin{Shaded}
\begin{Highlighting}[]
\NormalTok{comprobar_normalidad =}\StringTok{ }\ControlFlowTok{function}\NormalTok{(datos)\{}
  \ControlFlowTok{for}\NormalTok{(i }\ControlFlowTok{in} \DecValTok{1}\OperatorTok{:}\KeywordTok{ncol}\NormalTok{(datos))\{}
    \ControlFlowTok{if}\NormalTok{(}\KeywordTok{is.numeric}\NormalTok{(datos[,i]))\{}
      \KeywordTok{qqnorm}\NormalTok{(datos[,i], }\DataTypeTok{main =} \KeywordTok{paste}\NormalTok{(}\StringTok{"Normal Q-Q plot para "}\NormalTok{, }\KeywordTok{colnames}\NormalTok{(datos)[i]))}
      \KeywordTok{qqline}\NormalTok{(datos[,i], }\DataTypeTok{col=}\StringTok{"red"}\NormalTok{)}
      \KeywordTok{hist}\NormalTok{(datos[,i],}
           \DataTypeTok{main =} \KeywordTok{paste}\NormalTok{(}\StringTok{"Histograma para "}\NormalTok{, }\KeywordTok{colnames}\NormalTok{(datos)[i]),}
           \DataTypeTok{xlab =} \KeywordTok{colnames}\NormalTok{(datos)[i], }\DataTypeTok{freq =} \OtherTok{FALSE}\NormalTok{)}
\NormalTok{    \}}
\NormalTok{  \}}
\NormalTok{\}}

\NormalTok{trainIndex=}\KeywordTok{createDataPartition}\NormalTok{(dh_reducido}\OperatorTok{$}\NormalTok{hotel, }\DataTypeTok{p=}\FloatTok{0.06}\NormalTok{)}\OperatorTok{$}\NormalTok{Resample1}
\NormalTok{data_training=dh_reducido[trainIndex, ]}
\NormalTok{data_test=}\StringTok{ }\NormalTok{dh_reducido[}\OperatorTok{-}\NormalTok{trainIndex, ]}

\KeywordTok{par}\NormalTok{(}\DataTypeTok{mfrow=}\KeywordTok{c}\NormalTok{(}\DecValTok{2}\NormalTok{,}\DecValTok{2}\NormalTok{))}
\KeywordTok{comprobar_normalidad}\NormalTok{(data_training)}
\end{Highlighting}
\end{Shaded}

\includegraphics{prac2_files/figure-latex/normalidad-1.pdf}
\includegraphics{prac2_files/figure-latex/normalidad-2.pdf}
\includegraphics{prac2_files/figure-latex/normalidad-3.pdf}
\includegraphics{prac2_files/figure-latex/normalidad-4.pdf}
\includegraphics{prac2_files/figure-latex/normalidad-5.pdf}

Ahora para ver si las varibales estan normalizadas aplico el test de
Shapiro Wilk a cada variable numérica, para aplicarlo la muestra tienen
que ser inferior a 5000, Por eso reducimos la muestra anteriormente

\begin{Shaded}
\begin{Highlighting}[]
\KeywordTok{shapiro.test}\NormalTok{(data_training}\OperatorTok{$}\NormalTok{arrival_date_week_number)}
\end{Highlighting}
\end{Shaded}

\begin{verbatim}
## 
##  Shapiro-Wilk normality test
## 
## data:  data_training$arrival_date_week_number
## W = 0.96927, p-value < 2.2e-16
\end{verbatim}

\begin{Shaded}
\begin{Highlighting}[]
\KeywordTok{shapiro.test}\NormalTok{(data_training}\OperatorTok{$}\NormalTok{arrival_date_day_of_month)}
\end{Highlighting}
\end{Shaded}

\begin{verbatim}
## 
##  Shapiro-Wilk normality test
## 
## data:  data_training$arrival_date_day_of_month
## W = 0.95067, p-value < 2.2e-16
\end{verbatim}

\begin{Shaded}
\begin{Highlighting}[]
\KeywordTok{shapiro.test}\NormalTok{(data_training}\OperatorTok{$}\NormalTok{stays_in_weekend_nights)}
\end{Highlighting}
\end{Shaded}

\begin{verbatim}
## 
##  Shapiro-Wilk normality test
## 
## data:  data_training$stays_in_weekend_nights
## W = 0.78323, p-value < 2.2e-16
\end{verbatim}

\begin{Shaded}
\begin{Highlighting}[]
\KeywordTok{shapiro.test}\NormalTok{(data_training}\OperatorTok{$}\NormalTok{stays_in_week_nights)}
\end{Highlighting}
\end{Shaded}

\begin{verbatim}
## 
##  Shapiro-Wilk normality test
## 
## data:  data_training$stays_in_week_nights
## W = 0.81646, p-value < 2.2e-16
\end{verbatim}

\begin{Shaded}
\begin{Highlighting}[]
\KeywordTok{shapiro.test}\NormalTok{(data_training}\OperatorTok{$}\NormalTok{adults)}
\end{Highlighting}
\end{Shaded}

\begin{verbatim}
## 
##  Shapiro-Wilk normality test
## 
## data:  data_training$adults
## W = 0.67384, p-value < 2.2e-16
\end{verbatim}

\begin{Shaded}
\begin{Highlighting}[]
\KeywordTok{shapiro.test}\NormalTok{(data_training}\OperatorTok{$}\NormalTok{children)}
\end{Highlighting}
\end{Shaded}

\begin{verbatim}
## 
##  Shapiro-Wilk normality test
## 
## data:  data_training$children
## W = 0.29776, p-value < 2.2e-16
\end{verbatim}

\begin{Shaded}
\begin{Highlighting}[]
\KeywordTok{shapiro.test}\NormalTok{(data_training}\OperatorTok{$}\NormalTok{babies)}
\end{Highlighting}
\end{Shaded}

\begin{verbatim}
## 
##  Shapiro-Wilk normality test
## 
## data:  data_training$babies
## W = 0.071609, p-value < 2.2e-16
\end{verbatim}

\begin{Shaded}
\begin{Highlighting}[]
\KeywordTok{shapiro.test}\NormalTok{(data_training}\OperatorTok{$}\NormalTok{adr)}
\end{Highlighting}
\end{Shaded}

\begin{verbatim}
## 
##  Shapiro-Wilk normality test
## 
## data:  data_training$adr
## W = 0.94309, p-value < 2.2e-16
\end{verbatim}

El test de Shapiro Wilk nos indican que ninguna de las variables estan
normalizadas, ya que es p-valor es inferior a 0.05, asi que entendemos
que no es normal. De todas formas que no sea normal no significa que no
pueda llegar a serlo, ya que según el teorema del limite central al
tener mas de 30 elementos la observación podemos aproximarla como una
distribución normal de media 0 y desviacin estandar 1

\hypertarget{aplicaciuxf3n-de-pruebas-estaduxedsticas-para-comparar-los-grupos-de-datos.-en-funciuxf3n-de-los-datos-y-el-objetivo-del-estudio-aplicar-pruebas-de-contraste-de-hipuxf3tesis-correlaciones-regresiones-etc.-aplicar-al-menos-tres-muxe9todos-de-anuxe1lisis-diferentes.}{%
\paragraph{4.3. Aplicación de pruebas estadísticas para comparar los
grupos de datos. En función de los datos y el objetivo del estudio,
aplicar pruebas de contraste de hipótesis, correlaciones, regresiones,
etc. Aplicar al menos tres métodos de análisis
diferentes.}\label{aplicaciuxf3n-de-pruebas-estaduxedsticas-para-comparar-los-grupos-de-datos.-en-funciuxf3n-de-los-datos-y-el-objetivo-del-estudio-aplicar-pruebas-de-contraste-de-hipuxf3tesis-correlaciones-regresiones-etc.-aplicar-al-menos-tres-muxe9todos-de-anuxe1lisis-diferentes.}}

Para poder responder a las preguntas que hemos planteado anteriormente,
podemos usarpruebas de correlación lineal o regresión lineal, para ver
la relacion que existe entre dos variables, pero antes vamos a ver sus
diferencias: - La correlacion lo que hace es cuantificar como de
relacionadas estan dos variables - La regresión lineal lo que hace es
generar una ecuación, que pretende predecir el valor de una en funcion
de la otra.

Lo que se suele hacer es primeramente ver si ambas variables estan
relacionadas y en caso de estarlo se porcede a realizar el modelo de
regresión.

\begin{Shaded}
\begin{Highlighting}[]
\CommentTok{# A partir de ahora usaremos los datos de training}
\CommentTok{# por ejemplo queremos estudiar si hay una relacion entre el mes en el que se hace la reserva y el precio}
\CommentTok{# de la habitación.}

\CommentTok{# Haremos un diagrama de dispersión.}

\KeywordTok{ggplot}\NormalTok{(}\DataTypeTok{data =}\NormalTok{ data_training, }\KeywordTok{aes}\NormalTok{(}\DataTypeTok{x=}\NormalTok{  arrival_date_month, }\DataTypeTok{y=}\NormalTok{ adr)) }\OperatorTok{+}
\StringTok{  }\KeywordTok{geom_point}\NormalTok{(}\DataTypeTok{colour=} \StringTok{"red4"}\NormalTok{) }\OperatorTok{+}
\StringTok{  }\KeywordTok{ggtitle}\NormalTok{(}\StringTok{"Diagrama de dispersión")}
\end{Highlighting}
\end{Shaded}

\includegraphics{prac2_files/figure-latex/pruebas-1.pdf}

\begin{Shaded}
\begin{Highlighting}[]
\NormalTok{cor1 =}\StringTok{ }\NormalTok{data_training[, }\KeywordTok{c}\NormalTok{(}\StringTok{"arrival_date_month"}\NormalTok{, }\StringTok{"adr"}\NormalTok{)]}
\NormalTok{correlacion1 =}\StringTok{ }\KeywordTok{round}\NormalTok{(}\KeywordTok{cor}\NormalTok{(cor1), }\DecValTok{1}\NormalTok{)}
\KeywordTok{corrplot}\NormalTok{(correlacion1, }\DataTypeTok{addCoef.col =} \StringTok{"black"}\NormalTok{)}
\end{Highlighting}
\end{Shaded}

\includegraphics{prac2_files/figure-latex/pruebas-2.pdf}

\begin{Shaded}
\begin{Highlighting}[]
\CommentTok{# si hay una relacion entre el dia del mes y el precio}
\CommentTok{# de la habitación.}

\KeywordTok{ggplot}\NormalTok{(}\DataTypeTok{data =}\NormalTok{ data_training, }\KeywordTok{aes}\NormalTok{(}\DataTypeTok{x =}\NormalTok{ arrival_date_day_of_month , }\DataTypeTok{y=}\NormalTok{ adr)) }\OperatorTok{+}
\StringTok{  }\KeywordTok{geom_point}\NormalTok{(}\DataTypeTok{colour=} \StringTok{"red4"}\NormalTok{) }\OperatorTok{+}
\StringTok{  }\KeywordTok{ggtitle}\NormalTok{(}\StringTok{"Diagrama de dispersión")}
\end{Highlighting}
\end{Shaded}

\includegraphics{prac2_files/figure-latex/pruebas-3.pdf}

\begin{Shaded}
\begin{Highlighting}[]
\NormalTok{cor2 =}\StringTok{ }\NormalTok{data_training[, }\KeywordTok{c}\NormalTok{(}\StringTok{"arrival_date_day_of_month"}\NormalTok{, }\StringTok{"adr"}\NormalTok{)]}
\NormalTok{correlacion2 =}\StringTok{ }\KeywordTok{round}\NormalTok{(}\KeywordTok{cor}\NormalTok{(cor2), }\DecValTok{1}\NormalTok{)}
\KeywordTok{corrplot}\NormalTok{(correlacion2, }\DataTypeTok{addCoef.col =} \StringTok{"black"}\NormalTok{)}
\end{Highlighting}
\end{Shaded}

\includegraphics{prac2_files/figure-latex/pruebas-4.pdf}

\begin{Shaded}
\begin{Highlighting}[]
\CommentTok{# si hay una relacion entre el número de personas al que se hace la reserva y}
\CommentTok{# el precio de la habitación.}

\KeywordTok{ggplot}\NormalTok{(}\DataTypeTok{data =}\NormalTok{ data_training, }\KeywordTok{aes}\NormalTok{(}\DataTypeTok{x =}\NormalTok{ adults , }\DataTypeTok{y=}\NormalTok{ adr)) }\OperatorTok{+}
\StringTok{  }\KeywordTok{geom_point}\NormalTok{(}\DataTypeTok{colour=} \StringTok{"red4"}\NormalTok{) }\OperatorTok{+}
\StringTok{  }\KeywordTok{ggtitle}\NormalTok{(}\StringTok{"Diagrama de dispersión")}
\end{Highlighting}
\end{Shaded}

\includegraphics{prac2_files/figure-latex/pruebas-5.pdf}

\begin{Shaded}
\begin{Highlighting}[]
\NormalTok{cor3 =}\StringTok{ }\NormalTok{data_training[, }\KeywordTok{c}\NormalTok{(}\StringTok{"adults"}\NormalTok{, }\StringTok{"adr"}\NormalTok{)]}
\NormalTok{correlacion3 =}\StringTok{ }\KeywordTok{round}\NormalTok{(}\KeywordTok{cor}\NormalTok{(cor3), }\DecValTok{1}\NormalTok{)}
\KeywordTok{corrplot}\NormalTok{(correlacion3, }\DataTypeTok{addCoef.col =} \StringTok{"black"}\NormalTok{)}
\end{Highlighting}
\end{Shaded}

\includegraphics{prac2_files/figure-latex/pruebas-6.pdf}

\begin{Shaded}
\begin{Highlighting}[]
\CommentTok{# si hay una relacion entre el número de de dias entre semana  y}
\CommentTok{# el precio de la habitación.}

\KeywordTok{ggplot}\NormalTok{(}\DataTypeTok{data =}\NormalTok{ data_training, }\KeywordTok{aes}\NormalTok{(}\DataTypeTok{x =}\NormalTok{ stays_in_week_nights , }\DataTypeTok{y=}\NormalTok{ adr)) }\OperatorTok{+}
\StringTok{  }\KeywordTok{geom_point}\NormalTok{(}\DataTypeTok{colour=} \StringTok{"red4"}\NormalTok{) }\OperatorTok{+}
\StringTok{  }\KeywordTok{ggtitle}\NormalTok{(}\StringTok{"Diagrama de dispersión")}
\end{Highlighting}
\end{Shaded}

\includegraphics{prac2_files/figure-latex/pruebas-7.pdf}

\begin{Shaded}
\begin{Highlighting}[]
\NormalTok{cor4 =}\StringTok{ }\NormalTok{data_training[, }\KeywordTok{c}\NormalTok{(}\StringTok{"stays_in_week_nights"}\NormalTok{, }\StringTok{"adr"}\NormalTok{)]}
\NormalTok{correlacion4 =}\StringTok{ }\KeywordTok{round}\NormalTok{(}\KeywordTok{cor}\NormalTok{(cor4), }\DecValTok{1}\NormalTok{)}
\KeywordTok{corrplot}\NormalTok{(correlacion4, }\DataTypeTok{addCoef.col =} \StringTok{"black"}\NormalTok{)}
\end{Highlighting}
\end{Shaded}

\includegraphics{prac2_files/figure-latex/pruebas-8.pdf}

Como podemos comprobar tanto graficamente como numericamente no existe
una correlacion muy fuerte entre ningunas de las variables Por ejemplo
vamos a estimar por minimos cuadrados que relación hay entre en numero
de adultos para el que se hace la reserva, el mes de llegada y el precio
medio de la noche

\begin{Shaded}
\begin{Highlighting}[]
\NormalTok{regresion =}\StringTok{ }\KeywordTok{lm}\NormalTok{(adr }\OperatorTok{~}\StringTok{ }\NormalTok{adults }\OperatorTok{+}\StringTok{ }\NormalTok{stays_in_week_nights }\OperatorTok{+}\StringTok{ }\NormalTok{stays_in_weekend_nights, }\DataTypeTok{data=}\NormalTok{ data_training)}
\NormalTok{resultado_regresion =}\StringTok{  }\KeywordTok{summary}\NormalTok{(regresion)}
\NormalTok{resultado_regresion}
\end{Highlighting}
\end{Shaded}

\begin{verbatim}
## 
## Call:
## lm(formula = adr ~ adults + stays_in_week_nights + stays_in_weekend_nights, 
##     data = data_training)
## 
## Residuals:
##      Min       1Q   Median       3Q      Max 
## -107.291  -31.062   -7.014   23.457  264.653 
## 
## Coefficients:
##                         Estimate Std. Error t value Pr(>|t|)    
## (Intercept)              41.1556     2.6925  15.285   <2e-16 ***
## adults                   32.5830     1.4119  23.078   <2e-16 ***
## stays_in_week_nights      0.3233     0.4260   0.759    0.448    
## stays_in_weekend_nights  -1.2638     0.8344  -1.515    0.130    
## ---
## Signif. codes:  0 '***' 0.001 '**' 0.01 '*' 0.05 '.' 0.1 ' ' 1
## 
## Residual standard error: 47.2 on 4507 degrees of freedom
## Multiple R-squared:  0.1075, Adjusted R-squared:  0.1069 
## F-statistic:   181 on 3 and 4507 DF,  p-value: < 2.2e-16
\end{verbatim}

Explicando un poco el modelo con estas 4 variables es capaz de aplicar
el 10\% de la varianza observada en el precio de la habitación. (siendo
el R ajustado =0.1069) cuanto más próximo a 1 también indica que el
modelo es capaz de explicar una gran proporción de la varianza en la
variable respuesta.

El P-valor del modelo es significativo (2.2e-16) por lo que podemos
decir que el modelo es útil y que exite una relación entre los
predictores y la variable respuesta (uno de los coeficientes es distinto
de 0)

La estimación puntual seria o b0 : 42.4259 La estimación que acompaña a
adults que es b1: 30.8362 La estimacion que acompaña a
stays\_in\_week\_nights que es b2: 1.0104 y la que acompaña a
stays\_in\_weekend\_nights es b3:-1.4415

La ecuacion de regresión que representa a este modelo sería El modelo
teórico seria Y = b0 + b1\emph{adults + b2}stays\_in\_week\_nights +
b3*stays\_in\_weekend\_nights

Por otro lado podemos ver el contraste de precio entre dependiendo del
tipo de hotel que tenomos (City Hotel) y (Resort Hotel)

En este caso consideramos dos muestras independientes y lo queremos
saberes si, la tarifa diaria promedio es diferente al de los hoteles en
la ciudad o los resort, por lo tanto hay hay que considerar dos muestras
independientes. En este caso tenemos una muestra aleatoria de tamaño n1
y otra muestra de tamaño n2.

En este caso nuestra hipótesis nula sería: H0: μ1 = μ2 y nuestra
hipótesis alternativa es la siguiente H1: μ1 /= μ2

No se nos da ninguna información sobre esta distribución, asi que hemos
de analizar si se cumple el teorema del limite central, Dado que la
muestra es suficientemente grande podemos suponer que la media
aritmetica de la muestra se puede aproximar a una distribución normal.

La media muestral tiene esperanza μ y varianza σ2/n, además por ser
combinación lineal de variables normales es a su vez Normal

\begin{Shaded}
\begin{Highlighting}[]
\NormalTok{city_hotel =}\StringTok{ }\NormalTok{data_training[data_training}\OperatorTok{$}\NormalTok{hotel }\OperatorTok{==}\StringTok{ "City Hotel"}\NormalTok{, }\StringTok{"adr"}\NormalTok{]}
\NormalTok{resort_hotel =}\StringTok{ }\NormalTok{data_training[data_training}\OperatorTok{$}\NormalTok{hotel }\OperatorTok{==}\StringTok{ "Resort Hotel"}\NormalTok{, }\StringTok{"adr"}\NormalTok{]}

\KeywordTok{t.test}\NormalTok{(city_hotel, resort_hotel, }\DataTypeTok{alternative=}\StringTok{"two.side"}\NormalTok{, }\DataTypeTok{var.equal=}\OtherTok{TRUE}\NormalTok{,}\DataTypeTok{conf.level=}\FloatTok{0.95}\NormalTok{)}
\end{Highlighting}
\end{Shaded}

\begin{verbatim}
## 
##  Two Sample t-test
## 
## data:  city_hotel and resort_hotel
## t = 10.13, df = 4509, p-value < 2.2e-16
## alternative hypothesis: true difference in means is not equal to 0
## 95 percent confidence interval:
##  12.34599 18.27120
## sample estimates:
## mean of x mean of y 
## 106.15004  90.84145
\end{verbatim}

Con un p-value = 2.2e-16 menor de 0.05 por lo tanto rechazamos la
hipotesis nula, osea hay una direrencia entre un hotel en el ciudad y un
resort. Por lo tanto hay diferencias significativas entre las medias.
Podemos concluir que la media entre un hotel en la ciudad y uno resort
son distintas en cuanto al tarifa diaria.

\hypertarget{representaciuxf3n-de-los-resultados-a-partir-de-tablas-y-gruxe1ficas.}{%
\subsection{5. Representación de los resultados a partir de tablas y
gráficas.}\label{representaciuxf3n-de-los-resultados-a-partir-de-tablas-y-gruxe1ficas.}}

En este apartado hemos los graficos los hemos ido poniendo a medida que
ibamos desarrollando la practica para tener en la medida de lo posible
vista grafica y numerica de los resultados obtenidos.

\hypertarget{resoluciuxf3n-del-problema.-a-partir-de-los-resultados-obtenidos-cuuxe1les-son-las-conclusiones-los-resultados-permiten-responder-al-problema}{%
\subsection{6. Resolución del problema. A partir de los resultados
obtenidos, ¿cuáles son las conclusiones? ¿Los resultados permiten
responder al
problema?}\label{resoluciuxf3n-del-problema.-a-partir-de-los-resultados-obtenidos-cuuxe1les-son-las-conclusiones-los-resultados-permiten-responder-al-problema}}

En nuestro caso no pudimos obtener una relacion lineal entre las
variables peso y ninguna de las otras variables en el dataset,
estudiamos la correlacion entre cada una de ellas, pero los valores
obtenidos fueron muy cercanos al cero en todos los casos, por lo tanto
no encontramos una relacion fuerte entre las variables.

Por otro lado el analisis realizado sobre el tipo de hotel y tarifa
diaria, pudimos concluir que hay bastante diferencia de precio
dependiendo del tipo de hotel.

\end{document}
